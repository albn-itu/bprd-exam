\section{Opgave 3}
\subsection{Delopgave 1}\label{ass:3-1}
\subsubsection{Koden}
\begin{lstlisting}[language=fsharp]
type typ =
  ...
  | TypT of typ * int option         (* Tuple type                  *)
                                                                   
and access =                                                       
  ...
  | TupIndex of access * expr        (* Tuple indexing              *)
\end{lstlisting}

\subsubsection{Eksempel}
Output fra FSI:
\begin{lstlisting}
> open Absyn;;
> TypT (TypI, Some 2);;
val it : typ = TypT (TypI, Some 2)

> TypT (TypI, None);;
val it : typ = TypT (TypI, None)

> TupIndex (AccVar "t1", CstI 0);;
val it : access = TupIndex (AccVar "t1", CstI 0)
\end{lstlisting}

\subsubsection{Forklaring}
De to tilføjelser er lavet efter opgavens meget detaljerede beskrivelse.

\subsection{Delopgave 2}\label{ass:3-2}
\subsubsection{Koden}
\begin{lstlisting}[language=fsharp]
rule Token = parse
  ...
  | "(|"            { LPARBAR }  
  | "|)"            { RPARBAR }  
  | '('             { LPAR }
\end{lstlisting}

\subsubsection{Forklaring}
Tilføjelsen er lavet efter opgavens meget detaljerede beskrivelse.

\subsection{Delopgave 3}\label{ass:3-3}
\subsubsection{Koden}
\begin{lstlisting}[language=fsharp]
Vardesc: 
  ...
  | Vardesc LPARBAR RPARBAR             { compose1 (fun t -> TypT(t, None)) $1    }
  | Vardesc LPARBAR CSTINT RPARBAR      { compose1 (fun t -> TypT(t, Some $3)) $1 }
;

Access:
  ...
  | Access LPARBAR Expr RPARBAR         { TupIndex($1, $3)    }   
\end{lstlisting}

\subsubsection{Kompilering}
Jeg kompilerer med en række scripts som jeg har skrevet. De kan ses herunder.

\li{compile.sh}
\lstinputlisting[language=bash]{exercise-3/compile.sh}

\li{compileLexer.sh}
\lstinputlisting[language=bash]{../bin/compileLexer.sh}

\li{compileParser.sh}
\lstinputlisting[language=bash]{../bin/compileParser.sh}

\li{run.sh}
\lstinputlisting[language=bash]{../bin/run.sh}

\subsubsection{Eksempel}
Output fra FSI:
\begin{lstlisting}
> open ParseAndComp;;
> fromString "void main() {int t1(|2|);}";;
val it : Absyn.program =
  Prog [Fundec (None, "main", [], Block [Dec (TypT (TypI, Some 2), "t1")])]

> fromString "void main() {int t1(||);}";;
val it : Absyn.program =
  Prog [Fundec (None, "main", [], Block [Dec (TypT (TypI, None), "t1")])]

> fromString "void main() {t1(|0|) = 55;}";;
val it : Absyn.program =
  Prog
    [Fundec
       (None, "main", [],
        Block [Stmt (Expr (Assign (TupIndex (AccVar "t1", CstI 0), CstI 55)))])]

> fromString "void main() {print t1(|0|);}";;
val it : Absyn.program =
  Prog
    [Fundec
       (None, "main", [],
        Block
          [Stmt
             (Expr (Prim1 ("printi", Access (TupIndex (AccVar "t1", CstI 0)))))])]
\end{lstlisting}

\subsubsection{Forklaring}
Parser tilføjelserne er lavet efter opgavens meget detaljerede beskrivelse.

\subsection{Delopgave 4}\label{ass:3-4}
\subsubsection{Koden}
\begin{lstlisting}[language=fsharp]
let allocate (kind : int -> var) (typ, x) (varEnv : varEnv) : varEnv * instr list =
    let (env, fdepth) = varEnv 
    match typ with
    ...
    | TypA (TypT _, _) -> 
      raise (Failure "allocate: array of tuples not permitted")
    ...
    | TypT (TypT _, _) -> 
      raise (Failure "allocate: tuple of tuples not permitted")
    | TypT (TypA _, _) -> 
      raise (Failure "allocate: tuple of arrays not permitted")
    | TypT (t, Some i) ->
      let newEnv = ((x, (kind (fdepth), typ)) :: env, fdepth+i)  
      let code = [INCSP i;]
      (newEnv, code)

and cAccess access varEnv funEnv : instr list =
    match access with 
    ...
    | TupIndex(acc, idx) -> cAccess acc varEnv funEnv 
                            @ cExpr idx varEnv funEnv @ [ADD]
\end{lstlisting}

\subsubsection{Eksempel}
\begin{lstlisting}
# ./compile.sh
...

> open ParseAndComp;;
> compile "tuple";;
> #quit;;

# java Machine tuple.out
55 56 55 56
Ran 0.0 seconds
\end{lstlisting}

\subsubsection{Forklaring}
Løsningen er, som resten af opgaven, baseret på liste implementationen. Første tilføjelse er at \li{TypA} ikke kan instantieres med \li{TypT} elementer. De næste to tilføjelser er samme ide, \li{TypT} kan ikke instantieres med sin egen type eller \li{TypA}. Det er ikke understøttet da der bruges ret simpel indeksering ind i hukkomelsen som ikke ville kunne håndtere at skulle bruge et offset for hver liste i liste.\\[1ex]

Den sidste tilføjelse er instantiering af \li{TypT} med en konstant størrelse. Den skiller sig ud fra liste implementationen da der ikke er en pointer til det første element. Så variablen der tilføjes til miljøet er lokationen af det første element, i modsætning til en pointer til de første element. Derfor er instantierings instruktionerne også ret simple, da de bare udvider stacken med den størrelsen. Tilføjelsen i \li{cAccess} reflekterer den ændring ved ikke at lave en \li{LDI} efter \li{cAccess} evalueringen. Det er fordi resultatet af \li{cAccess} er lokationen af det første element, så den følgende addition af indekset er direkte derpå.

\subsection{Delopgave 5}\label{ass:3-5}
\subsubsection{Forklaring}
Ændringerne er forklaret i de opgaver hvor de redigeres i.

\subsection{Delopgave 6}\label{ass:3-6}
\subsubsection{Forklaring}
Eksemplet fra \ref{ass:3-4} duplikeres her.

\subsubsection{Eksempel}
\begin{lstlisting}
# ./compile.sh
...

> open ParseAndComp;;
> compile "tuple";;
> #quit;;

# java Machine tuple.out
55 56 55 56
Ran 0.0 seconds
\end{lstlisting}

