\section{Opgave 4}
\subsection{Delopgave }\label{ass:4-1}
\subsubsection{Koden}
\li{FunLex.fsl}
\begin{lstlisting}[language=fsharp]
rule Token = parse
  ...
  | '['             { LBRACK }
  | ']'             { RBRACK }
  | ','             { COMMA }
  | '@'             { AT }
\end{lstlisting}

\li{FunPar.fsy}
\begin{lstlisting}[language=fsharp]
...
\%token PLUS MINUS TIMES DIV MOD AT
\%token LPAR RPAR LBRACK RBRACK
\%token COMMA
...

\%left ELSE              /* lowest precedence  */
\%left EQ NE 
\%left GT LT GE LE
\%left PLUS MINUS AT
\%left TIMES DIV MOD 
\%nonassoc NOT           /* highest precedence  */

Expr:
  ...
  | Expr PLUS  Expr                     { Prim("+",  $1, $3)     }
  | Expr AT    Expr                     { Prim("@",  $1, $3)     }
  ...
;

AtExpr:
  ...
  | LBRACK List RBRACK                  { List($2)               }
;


List:
  Expr                                 { [$1]                   }
  | Expr COMMA List                    { $1 :: $3               }
;
\end{lstlisting}

\li{Absyn.fs}
\begin{lstlisting}[language=fsharp]
type expr = 
  ...
  | List of expr list
\end{lstlisting}

\subsubsection{Eksempel}
Output fra FSI:
\begin{lstlisting}
> open Parse;;
> fromFile "ex01.txt";;
val it : Absyn.expr =
  Let
    ("l1", List [CstI 2; CstI 3],
     Let
       ("l2", List [CstI 1; CstI 4],
        Prim
          ("=", Prim ("@", Var "l1", Var "l2"),
           List [CstI 2; CstI 3; CstI 1; CstI 4])))

> fromFile "ex02.txt";;
System.Exception: parse error in file ex02.txt near line 1, column 10
...

> fromFile "ex03.txt";;
val it : Absyn.expr =
  Let
    ("l", List [CstI 43],
     Prim ("@", Var "l", List [Prim ("+", CstI 3, CstI 4)]))

> fromFile "ex04.txt";;
val it : Absyn.expr =
  Let
    ("l", List [CstI 3],
     Prim
       ("=", Prim ("@", Var "l", List [CstI 3]),
        List [Prim ("+", CstI 3, CstI 4)]))

> fromFile "ex05.txt";;
val it : Absyn.expr =
  Letfun ("f", "x", Prim ("+", Var "x", CstI 1), List [Var "f"])

> fromFile "ex06.txt";;
val it : Absyn.expr = Letfun ("id", "x", Var "x", List [Var "id"])
\end{lstlisting}

\subsubsection{Forklaring}
Koden handler bare om at registrere symbolerne $[],@$ og så parse dem som lister, intent nyt her. ex02 fejler fordi opgaven specifikt siger "En ikke tom liste", så den kan ikke parse den tomme liste.

\subsection{Delopgave }\label{ass:4-2}
\subsubsection{Koden}
\begin{lstlisting}[language=fsharp]
type value = 
  | Int of int
  | Closure of string * string * expr * value env       (* (f, x, fBody, fDeclEnv) *)
  | ListV of value list

let rec eval (e : expr) (env : value env) : value =
    match e with
    ...
    | List l -> ListV (List.map (fun e -> eval e env) l) 
    | Prim(ope, e1, e2) -> 
      let v1 = eval e1 env
      let v2 = eval e2 env
      match (ope, v1, v2) with
      ...
      | ("=", ListV l1, ListV l2) -> Int (if l1 = l2 then 1 else 0)
      | ("@", ListV l1, ListV l2) -> ListV (l1 @ l2)
      ...
    ...
\end{lstlisting}

\subsubsection{Eksempel}
Output fra FSI:
\begin{lstlisting}
> open ParseAndRunHigher;;
> open Parse;;
> run(fromFile "ex01.txt");;
val it : HigherFun.value = Int 1

> run(fromFile "ex02.txt");;
System.Exception: parse error in file ex02.txt near line 1, column 10
...

> run(fromFile "ex03.txt");;
val it : HigherFun.value = ListV [Int 43; Int 7]

> run(fromFile "ex04.txt");;
val it : HigherFun.value = Int 0

> run(fromFile "ex05.txt");;
val it : HigherFun.value =
  ListV [Closure ("f", "x", Prim ("+", Var "x", CstI 1), [])]

> run(fromFile "ex06.txt");;
val it : HigherFun.value = ListV [Closure ("id", "x", Var "x", [])]

\end{lstlisting}

\subsubsection{Forklaring}
Først tilføjes typen til \li{value} som opgaven siger. Derefter tilføjes en constructor i \li{eval}, den mapper over expressions for at evaluere til deres \li{value} form. I \li{Prim} tilføjes et \li{=} case, som bare bruger FSharps indbyggede lighed på lister, som opfylder opgavens specifikation. Til sidst tilføjes en \li{@} operator som også bruger den interne version, da de fungerer på samme måde.


\subsection{Delopgave 3}\label{ass:4-3}
\subsubsection{Koden}
\begin{lstlisting}[language=fsharp]

\end{lstlisting}

\subsubsection{Eksempel}
Output fra FSI:
\begin{lstlisting}

\end{lstlisting}

\subsubsection{Forklaring}


