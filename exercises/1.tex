\section{Opgave 1}
\subsection{Delopgave 1}\label{ass:1-1}
\subsubsection{Koden}
\begin{lstlisting}[language=fsharp]
let numbers = FromTo(5,12);

let exam1 = Every(Write(numbers));
\end{lstlisting}
\subsubsection{Eksempel}
Output fra FSI:
\begin{lstlisting}
> run exam1;;
5 6 7 8 9 10 11 12 val it : value = Int 0
\end{lstlisting}

\subsubsection{Forklaring}
Koden skal læses indefra og ud. Så først laves en liste med 8 tal, da \li{FromTo} er inklusiv. Hvert tal pakkes ind i et \li{Write} så continuationen nu er en sekvens af 8 \li{Write} udtryk med et tal i. Hvis koden kørtes nu ville det kun være 5 der printes. For at eksekvere alle udtrykkene pakkes det ind i et \li{Every} udtryk som eksekverer det hele.

\subsection{Delopgave 2}\label{ass:1-2}
\subsubsection{Koden}
\begin{lstlisting}[language=fsharp]
let numbers = FromTo(5,12);

let exam2 = Every(Write(Prim("<", CstI 10, numbers)));
\end{lstlisting}

\subsubsection{Eksempel}
Output fra FSI:
\begin{lstlisting}
> run exam2;;
11 12 val it : value = Int 0
\end{lstlisting}

\subsubsection{Forklaring}
Koden minder meget om den fra \ref{ass:1-1}. Den eneste forskel er at \li{numbers} pakkes ind i et \li{Prim} udtryk der sammenligner det med 10 og kun printer tallene hvis de er over 10. Da 10 tallet placeres først i sammenligningen så er det $10<5,10<6,\ldots$  der sammenlignes, hvilket betyder at det kun er tallene højere end 10 der evaluerer sandt.

\subsection{Delopgave 3}\label{ass:1-3}
\subsubsection{Koden}
\begin{lstlisting}[language=fsharp]
let numbers = FromTo(5,12);

let exam3 = Every(Write(Prim("<", numbers, And(Write (CstS "\n"), numbers))));
\end{lstlisting}

\subsubsection{Eksempel}
Output fra FSI:
\begin{lstlisting}
> run exam3;;

 6 7 8 9 10 11 12
 7 8 9 10 11 12
 8 9 10 11 12
 9 10 11 12
 10 11 12
 11 12
 12
 val it : value = Int 0
\end{lstlisting}

\subsubsection{Forklaring}
Koden er en kombination af de tidligere opgaver. I det andet argument af \li{Prim} gives nu sekvensen. Mens i det tredje argument gives et \li{And} udtryk som evaluerer begge af sine argumenter men kun returnerer det andet. Så her printer koden først en ny linje med ny linje karakteren \li{\\n}, derefter returnerer den \li{numbers} sekvensen som vi kender den. Selve sammenligningen bliver nu til $8\cdot8$ sammenligninger, da vi sammenligner alle elementer i den sidste sekvens med $5$, printer en ny linje, sammenligner med $6$ osv til og med $12$. Dermed får vi en sekvens af tal hvor der fjernes et ved hver iteration. $5$ vises aldrig da $5<5$ ikke er sandt, det sidste $12$ tal er sammenligningen $11<12$, så der er en sidste tom linje der bare ikke vises. En dårlig ting ved strategien er den ekstra linje i toppen af outputtet, og at \li{val it : ...} vises på en ny linje. Den første er fordi \li{Write} evalueres før sekvensen, den sidste er fordi er er en tom linje.

\subsection{Delopgave 4}\label{ass:1-4}
\subsubsection{Koden}
Ændringer i \li{Icon.fs}
\begin{lstlisting}[language=fsharp]
type expr = 
  ...
  | FromToChar of char * char
  | Fail;;

let rec eval (e : expr) (cont : cont) (econt : econt) = 
    match e with
    ...
    | FromToChar(c1, c2) ->
      let rec loop c =
        if c <= c2 then
          cont (Str (c |> string)) (fun () -> loop ((c |> int) + 1 |> char))
        else
          econt ()

      loop c1
    | Fail -> econt ()

let chars = FromToChar('C', 'L');
let exam4 = Every(Write(chars));
\end{lstlisting}

\subsubsection{Eksempel}
Output fra FSI:
\begin{lstlisting}
> run exam4;;
C D E F G H I J K L val it : value = Int 0

> run chars;;
val it : value = Str "C"

> run (FromToChar('D', 'A'));;
Failed
val it : value = Int 0

> run (Every(Write(FromToChar('D', 'c'))));;
D E F G H I J K L M N O P Q R S T U V W X Y Z [ \ ] ^ _ ` a b c val it : value = Int 0
\end{lstlisting}

\subsubsection{Forklaring}
Koden er baseret på \li{FromTo} generatoren, samme løkke mekanik kan bruges da \li{char} typen kan sammenlignes ligesom en \li{int} i FSharp. Dog skal der lidt konvertering til at gå op i tegntabellen. Det gøres ved at oversætte \li{char} til \li{int} - \li{c |> int}, der adderes 1 og derefter konverteres tilbage til \li{char} - \li{...+ 1 |> char}. Se eksempel herunder. En del af opgaven var at generatoren laver \li{Str} typen, som tager en \li{string}. Det kan opnås ved at konveretere \li{char} til \li{string} således \li{c |> string}, som så pakkes ind i \li{Str} typen.

\begin{lstlisting}[language=fsharp]
> ('D' |> int) + 1 |> char;;
val it : char = 'E'
\end{lstlisting}

En ting der er værd at notere er at store og små bogstaver ikke ligger lige efter hinanden i tegntabellen så man kan ikke generere små til store, kun store til små, og hvis man gør så får man tegnene \li{[ \ ] ^ _ `} imellem tegnene.


\subsection{Delopgave 5}\label{ass:1-5}
\subsubsection{Koden}
\begin{lstlisting}[language=fsharp]
let rec eval (e : expr) (cont : cont) (econt : econt) = 
    match e with
    ...
    | Prim(ope, e1, e2) -> 
      eval e1 (fun v1 -> fun econt1 ->
          eval e2 (fun v2 -> fun econt2 -> 
              match (ope, v1, v2) with
              ...
              | ("<", Str s1, Str s2) -> 
                  if s1<s2 then 
                      cont (Str s2) econt2
                  else
                      econt2 ()

              | _ -> Str "unknown prim2")
              econt1)
          econt
    ...

let exam5_1 = Prim("<", CstS "A", CstS "B");
let exam5_2 = Prim("<", CstS "B", CstS "A");
\end{lstlisting}

\subsubsection{Eksempel}
Output fra FSI:
\begin{lstlisting}
> run exam5_1;;
val it : value = Str "B"

> run exam5_2;;
Failed
val it : value = Int 0
\end{lstlisting}

\subsubsection{Forklaring}
Løsningen er ret simpel, det er en kopi af sammenligningen af \li{Int} ændret til at tage \li{Str} i stedet. I FSharp er strenge sammenlignlige ligesom integers så der skal ikke ændres meget.

\subsection{Delopgave 6}\label{ass:1-6}
\subsubsection{Koden}
\begin{lstlisting}[language=fsharp]
let chars = FromToChar('C', 'L');

let exam6 = Every(Write(Prim("<", CstS "G", chars)));
\end{lstlisting}

\subsubsection{Eksempel}
Output fra FSI:
\begin{lstlisting}
> run exam6;;
H I J K L val it : value = Int 0
\end{lstlisting}

\subsubsection{Forklaring}
Løsningen er en simpel afart af \ref{ass:1-2}. Istedet for integers anvendes den nye sammenligning der understøtter strenge og så skal vi bare have alle tegn over \li{'G'}.
