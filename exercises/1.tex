\section{Opgave 1}
\subsection{Delopgave 1}\label{ass:1-1}
\subsubsection{Koden}
\begin{lstlisting}[language=fsharp]
let numbers = FromTo(5,12);

let exam1 = Every(Write(numbers));
\end{lstlisting}
\subsubsection{Eksempel}
Output fra FSI:
\begin{lstlisting}
> run exam1;;
5 6 7 8 9 10 11 12 val it : value = Int 0
\end{lstlisting}

\subsubsection{Forklaring}
Koden skal læses indefra og ud. Så først laves en liste med 8 tal, da \li{FromTo} er inklusiv. Hvert tal pakkes ind i et \li{Write} så continuationen nu er en sekvens af 8 \li{Write} udtryk med et tal i. Hvis koden kørtes nu ville det kun være 5 der printes. For at eksekvere alle udtrykkene pakkes det ind i et \li{Every} udtryk som eksekverer det hele.

\subsection{Delopgave 2}\label{ass:1-2}
\subsubsection{Koden}
\begin{lstlisting}[language=fsharp]
let numbers = FromTo(5,12);

let exam2 = Every(Write(Prim("<", CstI 10, numbers)));
\end{lstlisting}

\subsubsection{Eksempel}
Output fra FSI:
\begin{lstlisting}
> run exam2;;
11 12 val it : value = Int 0
\end{lstlisting}

\subsubsection{Forklaring}
Koden minder meget om den fra \ref{ass:1-1}. Den eneste forskel er at \li{numbers} pakkes ind i et \li{Prim} udtryk der sammenligner det med 10 og kun printer tallene hvis de er over 10. Da 10 tallet placeres først i sammenligningen så er det $10<5,10<6,\ldots$  der sammenlignes, hvilket betyder at det kun er tallene højere end 10 der evaluerer sandt.

\subsection{Delopgave 3}\label{ass:1-3}
\subsubsection{Koden}
\begin{lstlisting}[language=fsharp]
let numbers = FromTo(5,12);

let exam3 = Every(Write(Prim("<", numbers, And(Write (CstS "\n"), numbers))));
\end{lstlisting}

\subsubsection{Eksempel}
Output fra FSI:
\begin{lstlisting}
> run exam3;;

 6 7 8 9 10 11 12
 7 8 9 10 11 12
 8 9 10 11 12
 9 10 11 12
 10 11 12
 11 12
 12
 val it : value = Int 0
\end{lstlisting}

\subsubsection{Forklaring}
Koden er en kombination af de tidligere opgaver. I det andet argument af \li{Prim} gives nu sekvensen. Mens i det tredje argument gives et \li{And} udtryk som evaluerer begge af sine argumenter men kun returnerer det andet. Så her printer koden først en ny linje med ny linje karakteren \li{\n}, derefter returnerer den \li{numbers} sekvensen som vi kender den. Selve sammenligningen bliver nu til $8\cdot8$ sammenligninger, da vi sammenligner alle elementer i den sidste sekvens med $5$, printer en ny linje, sammenligner med $6$ osv til og med $12$. Dermed får vi en sekvens af tal hvor der fjernes et ved hver iteration. $5$ vises aldrig da $5<5$ ikke er sandt, det sidste $12$ tal er sammenligningen $11<12$, så der er en sidste tom linje der bare ikke vises.

\subsection{Delopgave 4}\label{ass:1-4}
\subsubsection{Koden}
\begin{lstlisting}[language=fsharp]

\end{lstlisting}

\subsubsection{Eksempel}
Output fra FSI:
\begin{lstlisting}

\end{lstlisting}

\subsubsection{Forklaring}



\subsection{Delopgave }\label{ass:1-}
\subsubsection{Koden}
\begin{lstlisting}[language=fsharp]

\end{lstlisting}

\subsubsection{Eksempel}
Output fra FSI:
\begin{lstlisting}

\end{lstlisting}

\subsubsection{Forklaring}


