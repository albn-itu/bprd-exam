\section{Opgave 2}
\subsection{Delopgave 1}\label{ass:2-1}
\subsubsection{Koden}
Kode ændringer givet her i diff format.
\begin{lstlisting}[language=diff]
diff --git a/exercise-2/CLex.fsl b/exercise-2/CLex.fsl
index 52e30e0..dc15b8b 100644
--- a/exercise-2/CLex.fsl
+++ b/exercise-2/CLex.fsl
@@ -35,6 +35,10 @@ let keyword s =
     | "true"    -> CSTBOOL 1
     | "void"    -> VOID 
     | "while"   -> WHILE         
+    | "createStack" -> CREATESTACK
+    | "pushStack" -> PUSHSTACK
+    | "popStack" -> POPSTACK
+    | "printStack" -> PRINTSTACK
     | _         -> NAME s
  
 let cEscape s = 
diff --git a/exercise-2/CPar.fsy b/exercise-2/CPar.fsy
index ed3a85f..3d6d73d 100644
--- a/exercise-2/CPar.fsy
+++ b/exercise-2/CPar.fsy
@@ -14,7 +14,7 @@ let nl = CstI 10
 %token <int> CSTINT CSTBOOL
 %token <string> CSTSTRING NAME
 
-%token CHAR ELSE IF INT NULL PRINT PRINTLN RETURN VOID WHILE
+%token CHAR ELSE IF INT NULL PRINT PRINTLN RETURN VOID WHILE CREATESTACK PUSHSTACK POPSTACK PRINTSTACK
 %token NIL CONS CAR CDR DYNAMIC SETCAR SETCDR
 %token PLUS MINUS TIMES DIV MOD
 %token EQ NE GT LT GE LE
@@ -134,6 +134,10 @@ ExprNotAccess:
   | Expr LE    Expr                     { Prim2("<=", $1, $3)   }
   | Expr SEQAND Expr                    { Andalso($1, $3)       }
   | Expr SEQOR  Expr                    { Orelse($1, $3)        }
+  | CREATESTACK LPAR Expr RPAR          { Prim1("createStack", $3) }
+  | PUSHSTACK LPAR Expr COMMA Expr RPAR { Prim2("pushStack", $3, $5) }
+  | POPSTACK LPAR Expr RPAR             { Prim1("popStack", $3) }
+  | PRINTSTACK LPAR Expr RPAR           { Prim1("printStack", $3) }
 ;
 
 AtExprNotAccess:
\end{lstlisting}

\subsubsection{Eksempel}
\begin{lstlisting}
# mono listcc.exe stack.lc
List-C compiler v 1.0.0.0 of 2012-02-13
Compiling stack.lc to stack.out
Prog
  [Fundec
     (None, "main", [],
      Block
        [Dec (TypD, "s");
         Stmt (Expr (Assign (AccVar "s", Prim1 ("createStack", CstI 3))));
         Stmt (Expr (Prim2 ("pushStack", Access (AccVar "s"), CstI 42)));
         Stmt (Expr (Prim2 ("pushStack", Access (AccVar "s"), CstI 43)));
         Stmt (Expr (Prim1 ("printStack", Access (AccVar "s"))));
         Stmt (Expr (Prim1 ("printi", Prim1 ("popStack", Access (AccVar "s")))));
         Stmt (Expr (Prim1 ("printi", Prim1 ("popStack", Access (AccVar "s")))));
         Stmt (Expr (Prim1 ("printStack", Access (AccVar "s"))))])]
\end{lstlisting}

\subsubsection{Forklaring}
Der skal en meget lille ændring til for at parse funktionerne. Først laves de som symboler i lexeren (\li{CLex.fsl}), derefter registreres de i parseren (\li{CPar.fsy}). Også i parser filen defineres deres format og output. Først defineres formatet ved at bruge symbolet, parentes symbolet (\li{LPAR}), udtryk specifikationen (\li{Expr}), og derefter luk parantes (\li{RPAR}). Herefter defineres output, i form af \li{Prim1} som gives en streng der kan matches på senere og det parsede argument. Det gentages for hver funktion, funktioner med 2 argumenter bruger komma symbolet (\li{COMMA}) mellem hvert argument og bruger \li{Prim2} istedet for \li{Prim1}.

\subsection{Delopgave 2}\label{ass:2-2}
\subsubsection{Koden}
Machine.fs:
\begin{lstlisting}[language=fsharp]
type instr =
  ...
  | CREATESTACK                        (* create stack                    *)
  | PUSHSTACK                          (* push to top of stack            *)
  | POPSTACK                           (* pop top of stack                *)
  | PRINTSTACK                         (* print stack                     *)

...
let CODESETCDR = 31
let CODECREATESTACK = 32
let CODEPUSHSTACK   = 33
let CODEPOPSTACK    = 34
let CODEPRINTSTACK  = 35;


let makelabenv (addr, labenv) instr = 
    match instr with
    ...
    | CREATESTACK    -> (addr+1, labenv)
    | PUSHSTACK      -> (addr+1, labenv)
    | POPSTACK       -> (addr+1, labenv)
    | PRINTSTACK     -> (addr+1, labenv)
\end{lstlisting}

listmachine.c:
\begin{lstlisting}[language=c]
...
#define STACKTAG 1
...

#define CREATESTACK 32
#define PUSHSTACK 33
#define POPSTACK 34
#define PRINTSTACK 35

void printInstruction(word p[], word pc) {
  switch (p[pc]) {
  ...
  case CREATESTACK:
    printf("CREATESTACK");
    break;
  case PUSHSTACK:
    printf("PUSHSTACK");
    break;
  case POPSTACK:
    printf("POPSTACK");
    break;
  case PRINTSTACK:
    printf("PRINTSTACK");
    break;
  ...
}

int execcode(word p[], word s[], word iargs[], int iargc,
             int /* boolean */ trace) {
  ...
    case CREATESTACK: {
      word n = Untag(s[sp]);

      if (n < 0) {
        printf("Cannot create a negative sized stack\n");
        return -1;
      }

      word *p = allocate(STACKTAG, n + 3, s, sp);
      s[sp] = (word)p; // Insert header

      p[1] = Tag(n); // Size
      p[2] = Tag(0); // Top
    } break;
    case PUSHSTACK: {
      word p = s[sp - 1];

      if (p == 0) {
        printf("Cannot push to null\n");
        return -1;
      }

      word *stack = (word *)p;
      word size = Untag(stack[1]);
      word top = Untag(stack[2]);

      if (top >= size) {
        printf("Cannot push to full stack\n");
        return -1;
      }

      word v = s[sp];
      stack[top + 3] = v;
      stack[2] = Tag(top + 1);
      sp--;
    } break;
    case POPSTACK: {
      word p = s[sp];

      if (p == 0) {
        printf("Cannot push to null\n");
        return -1;
      }

      word *stack = (word *)p;
      word top = Untag(stack[2]);

      if (top == 0) {
        printf("Cannot pop from empty stack");
        return -1;
      }

      s[sp] = stack[top + 2];
      stack[2] = Tag(top - 1);
    } break;
    case PRINTSTACK: {
      word p = s[sp];
      word *stack = (word *)p;

      word n = Untag(stack[1]);
      word top = Untag(stack[2]);
      printf("STACK(" WORD_FMT ", " WORD_FMT ")=[ ", n, top);
      for (int i = 0; i < top; i++) {
        printf(WORD_FMT " ", Untag(stack[i + 3]));
      }
      printf("]\n");
    } break;
    ...
}
\end{lstlisting}

\subsubsection{Eksempel}
Se \ref{ass:1-3}

\subsubsection{Forklaring}
I \li{Machine.fs} defineres instruktionerne med de definerede navne, derefter defineres deres instruktions nummer, det starver 32 og ender ved 35. Til sidst defineres label environment. Her sker der ikke noget specielt da instruktionerne selv ikke tager nogen argumenter, de tages fra stakken.\\[1ex]

\li{listmachine.c} starter også med at definere instruktionerne og deres nummer, samt et tag til stacks. \li{printInstruction} udvides med en streng version af instruktions navnet. Det spændende er \li{execcode}, her defineres håndteringen af instruktionerne. Først defineres \li{CREATESTACK}. Den finder størrelsen $N$ og allokerer hukkomelsen til stakken. Der allokeres som specifikationen siger, 3 ord mere end den adspurgte størrelse. Heri gemmes en header, størrelsen $N$ og hvor mange elementer der er indsat $top$. En pointer til stakken placeres på stakken.\\

\li{PUSHSTACK} henter stakken fra stakken og henter det element der skal pushes fra stakken. Derefter hentes top, og det nye element indsættes på stakken. Top øges med 1.\\

\li{POPSTACK} henter stakken fra stakken og henter top. Derefter hentes det øverste element fra stakken og placeres på stakken. Top nedsættes med 1. En ting der er værd at notere er at elementet hentes ved $top+2$ istedet for $top+3$ som normalt, da top er 1 foran det rent faktiske element.\\

\li{PRINTSTACK} henter stakken fra stakken og henter størrelsen og top. Derefter printes stakken ved at køre igennem fra $header\_lokation+3$ til $header\_lokation + (top-1)$. $-1$ af samme grund som $+2$ i \li{POPSTACK}.



\subsection{Delopgave 3}\label{ass:2-3}
\subsubsection{Koden}
\begin{lstlisting}[language=fsharp]
and cExpr (e : expr) (varEnv : varEnv) (funEnv : funEnv) : instr list = 
    match e with
    ...
    | Prim1(ope, e1) ->
      cExpr e1 varEnv funEnv
      @ (match ope with
         ...
         | "createStack" -> [CREATESTACK]
         | "popStack" -> [POPSTACK]
         | "printStack" -> [PRINTSTACK]

    | Prim2(ope, e1, e2) ->
      cExpr e1 varEnv funEnv
      @ cExpr e2 varEnv funEnv
      @ (match ope with
         ...
         | "pushStack" -> [PUSHSTACK]
\end{lstlisting}

\subsubsection{Eksempel}
Output fra FSI:
\begin{lstlisting}
# mono listcc.exe stack.lc
List-C compiler v 1.0.0.0 of 2012-02-13
Compiling stack.lc to stack.out
Prog
  [Fundec
  ...

# ./ListVM/ListVM/listmachine stack.out
STACK(3, 2)=[ 42 43 ]
43 42 STACK(3, 0)=[ ]

Used 0 cpu milli-seconds
\end{lstlisting}

\subsubsection{Forklaring}
Super simpel ændring. Instruktioner med 1 argument placeres i \li{Prim1} matches og mapper bare til sin egen instruktion. Instruktioner med 2 argumernter placeres ligeså i \li{Prim2}.


\subsection{Delopgave 4}\label{ass:2-4}
\subsubsection{Forklaring}
I forhold til tests så burde der være tests for success og fejl. Her er det altså når \li{createStack} får en negativ $N$ værdi, hvis \li{pushStack} får en 0 værdi som pointer eller hvis der ikke er plads på stakken. Det successfulde testcase er allerede lavet i \li{stack.lc}, så her kommer de fejlene.

\subsubsection{Koden}
\li{test1.lc}:
\begin{lstlisting}[language=c]
void main() {
  dynamic s;
  s = createStack(-1); // Fails here
}
\end{lstlisting}

\li{test2.lc}:
\begin{lstlisting}[language=c]
void main() {
  dynamic s;
  s = createStack(2);
  pushStack(s,42);
  pushStack(s,43);
  pushStack(s,43); // Fails here
  printStack(s);
}
\end{lstlisting}

\li{test3.lc}:
\begin{lstlisting}[language=c]
void main() {
  dynamic s;
  pushStack(s,42); // Fails here
}
\end{lstlisting}

\subsubsection{Eksempel}
\begin{lstlisting}
# mono listcc.exe test1.lc
# mono listcc.exe test2.lc
# mono listcc.exe test3.lc

# ./ListVM/ListVM/listmachine test1.out
Cannot create a negative sized stack

Used 0 cpu milli-seconds

# ./ListVM/ListVM/listmachine test2.out
Cannot push to full stack

Used 0 cpu milli-seconds

# ./ListVM/ListVM/listmachine test3.out
Cannot push to null

Used 0 cpu milli-seconds
\end{lstlisting}

