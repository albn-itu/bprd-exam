\section{Opgave 2}
\subsection{Delopgave 1}\label{ass:1-1}
\subsubsection{Koden}
Kode ændringer givet her i diff format.
\begin{lstlisting}[language=diff]
diff --git a/exercise-2/CLex.fsl b/exercise-2/CLex.fsl
index 52e30e0..dc15b8b 100644
--- a/exercise-2/CLex.fsl
+++ b/exercise-2/CLex.fsl
@@ -35,6 +35,10 @@ let keyword s =
     | "true"    -> CSTBOOL 1
     | "void"    -> VOID 
     | "while"   -> WHILE         
+    | "createStack" -> CREATESTACK
+    | "pushStack" -> PUSHSTACK
+    | "popStack" -> POPSTACK
+    | "printStack" -> PRINTSTACK
     | _         -> NAME s
  
 let cEscape s = 
diff --git a/exercise-2/CPar.fsy b/exercise-2/CPar.fsy
index ed3a85f..3d6d73d 100644
--- a/exercise-2/CPar.fsy
+++ b/exercise-2/CPar.fsy
@@ -14,7 +14,7 @@ let nl = CstI 10
 %token <int> CSTINT CSTBOOL
 %token <string> CSTSTRING NAME
 
-%token CHAR ELSE IF INT NULL PRINT PRINTLN RETURN VOID WHILE
+%token CHAR ELSE IF INT NULL PRINT PRINTLN RETURN VOID WHILE CREATESTACK PUSHSTACK POPSTACK PRINTSTACK
 %token NIL CONS CAR CDR DYNAMIC SETCAR SETCDR
 %token PLUS MINUS TIMES DIV MOD
 %token EQ NE GT LT GE LE
@@ -134,6 +134,10 @@ ExprNotAccess:
   | Expr LE    Expr                     { Prim2("<=", $1, $3)   }
   | Expr SEQAND Expr                    { Andalso($1, $3)       }
   | Expr SEQOR  Expr                    { Orelse($1, $3)        }
+  | CREATESTACK LPAR Expr RPAR          { Prim1("createStack", $3) }
+  | PUSHSTACK LPAR Expr COMMA Expr RPAR { Prim2("pushStack", $3, $5) }
+  | POPSTACK LPAR Expr RPAR             { Prim1("popStack", $3) }
+  | PRINTSTACK LPAR Expr RPAR           { Prim1("printStack", $3) }
 ;
 
 AtExprNotAccess:
\end{lstlisting}

\subsubsection{Eksempel}
\begin{lstlisting}
# mono listcc.exe stack.lc
List-C compiler v 1.0.0.0 of 2012-02-13
Compiling stack.lc to stack.out
Prog
  [Fundec
     (None, "main", [],
      Block
        [Dec (TypD, "s");
         Stmt (Expr (Assign (AccVar "s", Prim1 ("createStack", CstI 3))));
         Stmt (Expr (Prim2 ("pushStack", Access (AccVar "s"), CstI 42)));
         Stmt (Expr (Prim2 ("pushStack", Access (AccVar "s"), CstI 43)));
         Stmt (Expr (Prim1 ("printStack", Access (AccVar "s"))));
         Stmt (Expr (Prim1 ("printi", Prim1 ("popStack", Access (AccVar "s")))));
         Stmt (Expr (Prim1 ("printi", Prim1 ("popStack", Access (AccVar "s")))));
         Stmt (Expr (Prim1 ("printStack", Access (AccVar "s"))))])]
\end{lstlisting}

\subsubsection{Forklaring}
Der skal en meget lille ændring til for at parse funktionerne. Først laves de som symboler i lexeren (\li{CLex.fsl}), derefter registreres de i parseren (\li{CPar.fsy}). Også i parser filen defineres deres format og output. Først defineres formatet ved at bruge symbolet, parentes symbolet (\li{LPAR}), udtryk specifikationen (\li{Expr}), og derefter luk parantes (\li{RPAR}). Herefter defineres output, i form af \li{Prim1} som gives en streng der kan matches på senere og det parsede argument. Det gentages for hver funktion, funktioner med 2 argumenter bruger komma symbolet (\li{COMMA}) mellem hvert argument og bruger \li{Prim2} istedet for \li{Prim1}.

\subsection{Delopgave 2}\label{ass:1-2}
\subsubsection{Koden}
\begin{lstlisting}[language=fsharp]

\end{lstlisting}

\subsubsection{Eksempel}
Output fra FSI:
\begin{lstlisting}

\end{lstlisting}

\subsubsection{Forklaring}



\subsection{Delopgave }\label{ass:1-}
\subsubsection{Koden}
\begin{lstlisting}[language=fsharp]

\end{lstlisting}

\subsubsection{Eksempel}
Output fra FSI:
\begin{lstlisting}

\end{lstlisting}

\subsubsection{Forklaring}


